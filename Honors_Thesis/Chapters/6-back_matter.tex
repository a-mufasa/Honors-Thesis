\chapter{Broader Impacts of Research}

\section{Applicability of Research Methods to Other Problems}
The innovative techniques and technology utilized in the creation of the Insect Detection Server hold great potential for application in various fields beyond agriculture. The incorporation of Computer Vision and AI for insect identification and classification, along with GPT-3.5's ability to simulate agricultural expertise, can be tailored to tackle various issues in different industries.

\begin{enumerate}
   \item Medical Imaging: In diagnostic scans, the Computer Vision methods employed for insect detection can be applied to medical imaging to identify and categorize medical conditions, such as tumors or lesions
   \item Environmental Conservation: The technology for wildlife monitoring and conservation initiatives can be modified. Utilizing computer vision to identify and track endangered species in their natural habitats allows conservationists to gather essential data on population dynamics and devise effective conservation strategies.
   \item Security and Surveillance: The computer vision model can be employed in security and surveillance systems to detect and recognize suspicious activities or individuals, thus enhancing security measures in public spaces and private properties.
   \item Customer Service: The implementation of GPT-3.5 as an ``online expert" can be expanded to other areas, including customer service and technical support. Companies can utilize AI chatbots to offer instant and personalized customer assistance, boosting customer satisfaction and reducing response times.
\end{enumerate}

\section{Impact of Research Results on U.S. and Global Society}
The Insect Detection Server has the potential to generate substantial positive impacts on both U.S. and global society, particularly within the agricultural sector.

\begin{enumerate}
    \item Improved Pest Management: The capacity to accurately identify and classify insects on farms enables farmers to implement targeted and timely pest control measures. This can result in more effective pest management, decreased crop damage, and heightened agricultural productivity.

    \item Reduced Pesticide Use: By supplying farmers with precise information about pest presence, the technology can help minimize the indiscriminate application of pesticides. This can reduce the environmental and health hazards associated with pesticide exposure.

    \item Enhanced Food Security: Better pest management and decreased crop losses contribute to improved food security by increasing the availability and affordability of food. This is especially crucial in regions where agriculture is a primary source of livelihood, and food security is a significant concern.

    \item Access to Expertise: The integration of GPT-3.5 as an ``online expert" offers farmers expert advice and guidance on pest management. This democratizes access to agricultural knowledge, particularly for small-scale farmers who may not have access to traditional agricultural extension services.
\end{enumerate}

\section{Impact of Research Results on the Environment}
The implementation of the Insect Detection Server can produce several positive environmental impacts:

\begin{enumerate}
    \item Reduction in Pesticide Pollution: By facilitating targeted pest management, the technology can decrease pesticide usage, reducing pesticide runoff into water bodies and soil contamination. This helps safeguard aquatic and terrestrial ecosystems from the detrimental effects of pesticides.

    \item Biodiversity Conservation: By curbing the indiscriminate application of pesticides, the technology can help protect non-target species, including beneficial insects such as pollinators, from pesticide exposure. This supports the conservation of biodiversity in agricultural environments.

    \item Sustainable Agriculture: The technology promotes sustainable agriculture practices by providing farmers with data-driven insights for pest management. This empowers farmers to make informed decisions that balance agricultural productivity with environmental responsibility.
\end{enumerate}

In conclusion, the Insect Detection Server holds the potential to promote more sustainable and eco-friendly agricultural practices. By harnessing AI and computer vision technologies, the research results can lead to enhanced pest management, reduced environmental impacts, and agricultural sustainability.

\chapter{Identification of All Software Used in Research and Thesis/Dissertation Generation}

Machine \#1:\\
Owner: Ahmed Moustafa\\\\
\begin{tabular}{ll}
Software\\
\hline
Ubuntu 18.0.4 \\
TypeScript \\
Open source web-development libraries \\
Google Firebase \\
OpenAI API \\
Netlify \\\\
\end{tabular}

Machine \#2:\\
Included CPU on the prototype bug system \\\\
\begin{tabular}{ll}
Software\\ 
\hline
Ubuntu 16.04 \\
Required scripts for system operation \\
\end{tabular}



\chapter{Plagiarism Check}
This thesis was created by Ahmed Moustafa. I attest that all work found in this document is my own. Any examples of repetition in other works is incidental and not plagiarized material.
\vspace{1in}
\makebox[1.5in]{\hrulefill}\\
