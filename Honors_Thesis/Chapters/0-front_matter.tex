\title{Smart-Insect Monitoring System Integration and Interaction via \\
AI Cloud Deployment and GPT}

\author{Ahmed Moustafa}
\degreeyear{2023}

\degreetitle{Bachelor of Science} 

\field{Computer Science}
\chair{Dr. Khoa Luu}
\othermembers{
Dr. John Gauch, CSCE\\
Dr. Alexander Nelson, CSCE\\ 
}
\numberofmembers{3} 

\abstract{ 
The Insect Detection Server was developed to explore the deployment and integration of an Artificial Intelligence model for Computer Vision in the context of insect detection. The model was developed to accurately identify insects from images taken by camera systems installed on farms. The goal is to integrate the model into an easily accessible, cloud-based application that allows farmers to analyze automatically uploaded images containing groups of insects found on their farms. The application returns the bounding boxes and the detected classes of insects whenever an image is captured on-site, enabling farmers to take appropriate actions to address the issue of the insects' presence.
To extend the capabilities of the application, the server is linked to a GPT-3.5 API. This will allow users to ask questions about the bugs detected on their farms, creating an ``online expert"-like feature. Python, C++, and Computer Vision libraries were used for the detection model, while the OpenAI API was used for GPT-3.5's integration. By combining these technologies, farmers can more effectively and efficiently manage pests on their farms than current alternatives. This Generative Pre-trained Transformer (GPT) aspect of the project can be leveraged to enable the emulation of agricultural experts for users/farmers. The large language model (LLM) neural network can be fine-tuned using prompt engineering to generate natural language responses to user queries. This will enable farmers to get expert advice and guidance on pest management without having to consult with a human expert. The integration of GPT-3.5 API will also allow the application to provide personalized recommendations based on each farm's specific needs and circumstances. This added feature will give the farmers a more comprehensive and tailored approach to pest management, further increasing the efficiency and effectiveness of their pest control strategies.
The significance of this research lies in the development of a practical and accessible tool for farmers to manage pests on their farms. Using Computer Vision and Artificial Intelligence, farmers can quickly and accurately identify insects, leading to more efficient and effective pest management. This could help farmers reduce the use of pesticides and other forms of pest management, leading to improved crop yields and reduced environmental impacts. The potential benefits of this technology extend beyond the agricultural industry, as the techniques used in this research can be applied to a wide range of computer vision and user-facing data analytic applications. For example, the developed techniques could be applied to other fields, such as surveillance, security, and medical imaging.
}

\begin{frontmatter}
\makefrontmatter 

\begin{acknowledgements} 
I want to express my sincere gratitude to several individuals and institutions who have played a crucial role in the completion of this thesis. First and foremost, I would like to thank Dr. Khoa Luu, my honors research mentor, for his invaluable guidance, support, and mentorship throughout the entire research process. Dr. Luu's expertise, insights, and constructive feedback have been instrumental in shaping this work and in helping me develop as a researcher. I would also like to extend my thanks to Thanh-Dat Truong, the graduate student and fellow researcher in the Computer Vision and Image Understanding (CVIU) lab who generously shared his time and expertise to help with the model and project as a whole. Dat's contributions were essential to the success of this research, and I am very grateful for his support. I am also indebted to the University of Arkansas for providing the resources and opportunities necessary to facilitate this Undergraduate Honors Research. The university's commitment to fostering undergraduate research has been instrumental in my academic and professional development, and I am honored to have been a part of this program. Finally, I would like to thank my thesis defense committee members, Dr. Khoa Luu, Dr. John Gauch, and Dr. Alexander Nelson, for their time and expertise.
\end{acknowledgements}

\tableofcontents
\listoffigures
% \listoftables   % Uncomment if you have any tables TABLES 

\newpage


\end{frontmatter}